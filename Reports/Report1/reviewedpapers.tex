\section{Overview On Reviewed Papers}
Since the defined problem in 2012 TREC Contextual Suggestion Track is some how similar to the current problem, I focused on reviewing the solution proposed by participants of this Track. In TREC Contextual Suggestion Track, different factors like personal preferences, location and time influence on ranking of venues. I have tried to concentrate on the way different papers have encountered with the ranking problem in order to gain some ideas for current problem.

Here is a short brief of some of them:

\subsection{CSIRO ICT Centre \cite{CSIRO}} 
Besides the solution presented for accumulating the data of venues, it offers a vector-space approach for identifying personal preferences. Armed with a preference vector for each profile, it scores each venue according to its cosine distance.


\subsection{Georgetown University \cite{GeorgetownUniversity}}
The represented solution for ranking seems very simple and straightforward. First, it creates a list of venues in different categories. Then based on scores for each category which is filled in user's profile, it sums per-user category scores with the scores for every venues in each category. Finally it returns the first top ten venues. Boosting well-known attractions, using learning techniques (SVMRank) for ranking venues in each category and defining a limit for each set are some special points in the paper.

\subsection{ICTNET \cite{ICTNET}}
The final formula for ranking consisted of some fixed factors which are selected using feedback method. Although the formula seems very clear and practical, the way to achieve the factors is not explained clearly.


\subsection{IRIT \cite{IRIT}}
I found the paper quite interesting as it acquired the first ranking between other participants. 
