\section{Development}
\label{sec:development}
In this section, we describe some special issues during the development of the platform.

\subsection{Order}
The Order system is a full client-side web-page following the idea of SPA (Single Page Application) using KnockoutJS and BreezeJS. By applying MVVM model, the data submitted in each step is stored in View-Model and used in the next steps. The system screen-shots from the ordering process are shown in Figure \ref{fig:order-1} and Figure \ref{fig:order-2}.  The payment process is verified by a third-party service which supports different payment cards.

\subsection{Machine Learning's Data Accumulating}
In order to tackle the problem of ranking translators, a system similar to Order is designed. The system aims to accumulate data for a Learning To Rank process. Based on the query document, it offers three translators and stores the client's (evaluator's) choice. In contrast to the Order system, it eliminates the pictures and names of translators in order to reduce clients' bias in the final answer and puts emphasis on the presented criteria. A screen-shot of the system related to selecting the translators is shown in Figure \ref{fig:dataaccumulating}. In the next step, the data gathered by the system is applied on Learning to Rank algorithms which aim to discover the underlying pattern for proposing the best translators. The details are described in seminar lecture related to the project. 

\subsection{Text Matching}
Using Lucene, the documents of translators are categorized, indexed and stored based on their languages. In the next step, the similarity between each document and query document is calculated. In order to achieve the final text matching value, the calculated similarities should be aggregated. We use GP2\cite{gp2} algorithm to aggregate similarity values.

\subsection{Translation Memory}
As mentioned before, the aim of a Translation Memory system is helping the translators to exploit their previous translated texts in their new translations. In order to develop the system, first the HunAlign library is used for aligning the sentences of one text with the sentences of its corresponding text in the other language. While the algorithm tends to be error-prone in its basic version specially for complex texts, by adding dictionaries and using just rather simple and straightforward texts, reasonable aligned sentences were achieved. At the end, Lucene indexes the text and provides searching abilities.

\subsection{Scheduling}
The scheduling system is also designed by following the SPA idea and using pure Java-Script libraries and Web-API. Figure \ref{fig:scheduling} depicts the schedule web-page interface for an arbitrary translator.

\subsection{Chat}
Using Signal-R capabilities, we develop the Chat system. Related interfaces for meeting room as well as chatting with help-desk user are shown in Figure \ref{fig:chat}.


\begin{figure}[h]
\begin{center}
\includegraphics[scale=\figurescaling]{figures/order-1.jpg}
\caption{Order - entering primitive data
\label{fig:order-1}}
\includegraphics[scale=\figurescaling]{figures/order-2.jpg}
\caption{Order - selecting translator
\label{fig:order-2}}
\end{center}
\end{figure}

\begin{figure}[h]
\begin{center}
\includegraphics[scale=\figurescaling]{figures/dataaccumulation.jpg}
\caption{Data Accumulating - selecting translator
\label{fig:dataaccumulating}}
\includegraphics[scale=\figurescaling]{figures/chat.jpg}
\caption{Chat system
\label{fig:chat}}
\end{center}
\end{figure}


\begin{figure}[h]
\begin{center}
\includegraphics[scale=\figurescaling]{figures/scheduling.jpg}
\caption{Scheduling
\label{fig:scheduling}}
\end{center}
\end{figure}