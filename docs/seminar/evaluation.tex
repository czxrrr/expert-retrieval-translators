\section{Apply The Methods and Results}
\label{sec:apply}
In this section, we applied different approaches on the platform. By comparing the methods, we aim to discover the most appropriate one regarding to the project's characteristics and data.

\subsection{Aggregation Functions}
Similar to \cite{agg-gp2}, outcome of three algorithms are compared. Top1 and Top5 are selected as two common forms TopN aggregation algorithm. TopN refers to algorithm that summarizes the N top documents (i.e. Top1 only using the top associated document to rank the candidates).

Feedbacks of Proof-readers after every translation are used as a basis for evaluating the algorithms. Since feedbacks are a measure for quality of translation, the more similar the ranking of algorithms to feedbacks are the better it is.

The result of applying Spearman correlation is shown in Table\ref{table:Spearman}. As the outcome shows, GP2 outperforms the other algorithms. In comparison to Top1, Top5 has slightly better performance. The results is also partly the same when comparing based on language-pairs.

\begin{table}
\begin{center}
\begin{tabular}{|c|c|c|c|}
\hline Algorithm & Top1 & Top5 & GP2  \\
\hline Spearman correlation & 0.049 & 0.076 & 0.14\\
\hline
\end{tabular}
\caption{Comparison between algorithms and feedbacks}
\label{table:Spearman}
\end{center}
\end{table}

\subsection{Learning To Rank}
TODO