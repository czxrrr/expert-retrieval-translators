The goal of expertise retrieval is to link humans to expertise areas, and vice versa. In other words, the task of expertise retrieval is to identify a set of persons with relevant expertise for the given query \cite{er} \cite{er-community-aware}.

The launch of the Expert Finding task at TREC has generated a lot of interest in expertise retrieval, with rapid progress being made in terms of modeling, algorithms, and evaluation aspects.

With the development of information retrieval (IR) techniques, many research efforts in this field have been made to address high-level IR and not just traditional document retrieval, such as entity retrieval and expertise retrieval \citep{er-sparse}. Expertise retrieval has received increasing interest since the introduction of an expert finding task in the Text Retrieval Conference (TREC) 2005 \cite{trec2005} \cite{er-community-aware}.

Two principal approaches are proposed in \cite{trec2005} based on probabilistic language modeling techniques. They were formalized as so-called candidate models and document models. the candidate-based approach is also referred to as
profile-based method in builds a textual representation of candidate experts and then rank them based on the query. The document models' idea is to first find documents that are relevant to the topic and then locate the experts associated with these documents \cite{er}.

Ranking techniques are mostly one of the essential parts of IR frameworks. In recent years, Learning to Rank (L2R) has been studied extensively specially for document retrieval. It refers to machine learning techniques for training the model in a ranking task \cite{er}. In essence, expert search is a ranking problem and thus the existing L2R techniques can be naturally applied to it \cite{l2r-intro}.

As well as ranking techniques, aggregation functions have a solid effect on the performance of IR systems. Aggregate tasks are those where documents' similarities are not
the final outcome, but instead an intermediary component. In expert search, a ranking of candidate persons with relevant expertise to a query is generated after aggregation of their related documents \citep{agg-learning}.

This paper addresses the problem of searching translators as expertises. We have applied Learning to Rank in a candidate-based approach. Besides, different aggregation algorithms related to documents of translators has been studied.

The remaining of the paper is organized as follows. In Section \ref{sec:casestudy}, the Translator-Expert Retrieval framework is described in detail. Then, Section \ref{sec:methods} explains methods used in the study. In Section \ref{sec:apply}, we report the result of applied methods on the framework. Finally, we conclude the study in Section \ref{sec:conclusion}.
