\section{Introduction}
\label{sec:introduction}
We look at the technology of Information Retrieval from the perspective of a real-world user scenario involving the selection of human translators based on a combination of expertise and practical factors. 
It has become more and more common place to consider search technology in a series of applications previously served only by database technology, if at all by a computer system~\cite{Grefenstette:2011}. In such cases, new data, new users, and new scenarios need to be observed, existing methods have to be adapted to the task at hand, and new evaluation procedures have to be devised. 

This paper addresses the problem of searching translators as experts. We offer a novel translator-expert retrieval platform and evaluate different expert retrieval methods based on the practical implementation of the platform. In contrast to common expert retrieval systems, we include significant factors of searching a translator (such as price and delivery time) in the automated decision process as well as the relevance of the documents each translator has previously translated to the query document. The proposed method has two distinct components: A proficiency estimation phase, in which different aggregation algorithms related to documents of translators have been studied and compared; and a Learning-to-Rank phase in which different features are tested under all state of the art Learning-to-Rank methods based on a manually created ground truth, which we make available together with the source code of our application under GPL\footnote{\url{http://github/anonymized}}. 

The remaining of the paper is organized as follows. In Section\ref{sec:casestudy}, the use-case is presented and the Translator-Expert Retrieval framework is described in detail. Then, Section\ref{sec:methods} explains the methods used in the study. Section\ref{sec:apply} shows the result of applied methods on the framework. We discuss these results and conclude the study in Section\ref{sec:conclusion}.
