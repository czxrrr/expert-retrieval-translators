\section{Case Study}
\label{sec:casestudy}
The data flow of expert searching is depicted in Figure\ref{fig:architecture}. The client submits a document and searches for translators with a specific target language. Based on query document, the framework figures out offered price, delivery time, proficiency of translators and number of cooperation times related to each translator. The ranking system processes the calculated values and the offers the most related translators to the client.

As it is shown in Figure \ref{fig:architecture}, the essential components of the framework are \textit{Ranking}, \textit{Proficiency Estimator}, \textit{Scheduler} and \textit{Profiler}.

\begin{figure*}[h]
\begin{center}
\includegraphics[scale=0.8]{figures/dataflow.jpg}
\caption{System Data Flow
\label{fig:architecture}}
\end{center}
\end{figure*}

The \textit{Profiler} is responsible for accumulating personal information, translators' preferences, offered price and translation duration per word. The \textit{Ranking} system uses Learning to Rank techniques to return the most related translators to the client. The training data is provided by a group of annotators who are familiar with business of company using an evaluation system. The evaluation system suggests three translators and the annotators rate them by comparing between their factors. In order to prevent bias in evaluation, the translators are suggested randomly and without name and picture. Applied learning to rank methods and results are described in Section \ref{sec:apply}. The \textit{Proficiency Estimator} system stores the previous-translated documents in the cloud and indexes them using Lucene library. The similarity between query and indexed documents is used as a basis for estimation of translator's proficiency. In order to find proficiency value, the platform aggregates the similarity scores. The applied aggregation function is described in Section \ref{sec:apply}. The \textit{Scheduler} system figures out the delivery time based on timetable of translators. The scheduler builds a special data structure to calculate the delivery time in a reasonable response time. The detail of the process is out of scope of the paper.

Beside the translator, a proofreader selected by the client revises the final translation. As well as reviewing, the proofreader assesses the quality of translation from different points of view (grammar, style, accuracy, content and language). The assessment is defined as a value between $1$ (very bad) and $5$ (perfect). As it is discussed in Section \ref{sec:apply}, it is used to evaluate and compare aggregation algorithms for translator's proficiency.
